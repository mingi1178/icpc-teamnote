% twocolumn
% \documentclass[landscape, 8pt, a4paper, oneside, twocolumn]{extarticle}

\documentclass[landscape, 8pt, a4paper, oneside]{extarticle}
\usepackage{kotex}
\usepackage{amssymb}
\usepackage{amsmath}
\usepackage{import}
\usepackage{enumitem}
\usepackage{multicol}
\usepackage{titlesec}

\usepackage{teamnote}

\teamnote{University of Seoul}{GeongongiIsMae}{Junseo Park, Jimin Kim, Mingi Jeong}{2025 ICPC Seoul Regional (BEXCO, Busan)}

% smaller section title
% \titleformat{\subsection}{\normalfont\normalsize\bfseries}{\thesubsection}{1em}{}

\titlespacing*{\section}{0pt}{0pt}{0pt}
\titlespacing*{\subsection}{0pt}{0pt}{0pt}

\ShowUsage
\ShowComplexity
\HideAuthor

\begin{document}

\begin{multicols*}{3}

\maketitlepage

% break column
% \vfill\null\columnbreak

% Make Pagebreak if you want.
% \pagebreak

\section{Python}
\Algorithm{정렬 비교 함수 구현}{첫 번째가 두 번째보다 작으면(less-than) 음수 값(유지), 같으면 0, 크면(greater-than) 양수 값(교체)}{}{py}{code/Python/Sorting.py}{}
\Algorithm{itertools}{}{}{py}{code/Python/itertools.py}{}

\section{STL}
\Algorithm{set}{}{}{cpp}{code/STL/set.cpp}{}
\Algorithm{map}{}{}{cpp}{code/STL/map.cpp}{}
\Algorithm{priority\_queue}{}{}{cpp}{code/STL/priority_queue.cpp}{}
\Algorithm{bitset}{낭비되는 7bit를 없애기 위해서 0, 1을 하나 당 1bit에 저장하는 자료형}{}{cpp}{code/STL/bitset.cpp}{}
\Algorithm{permutation}{}{}{cpp}{code/STL/permutation.cpp}{}

\section{정렬/이분 탐색 관련}

\Algorithm{정렬 비교 함수 구현}{}{}{cpp}{code/Sorting/Compare.cpp}{JusticeHui}
\Algorithm{좌표 압축}{}{}{cpp}{code/Sorting/CoordCompress.cpp}{JusticeHui}
\Algorithm{이분 탐색 관련 STL}{}{}{cpp}{code/Sorting/STL.cpp}{JusticeHui}


\section{자료구조}

\Algorithm{PBDS}
{order\_of\_key(NUM) : ordered\_set 에서 NUM 보다 작은(미만의) 원소의 개수를 반환한다.\\
find\_by\_order(K) : ordered\_set 에서 (K+1)번째 원소가 있는 iterator 을 반환한다. (K가 0이면 1번째)}{}
{c}{code/DataStructure/pbds.cpp}{}

\TeamnoteImport{Erasable Priority Queue}{code/DataStructure/ErasablePQ.cpp}

\Algorithm{펜윅 트리(BIT)}
{}{$O(\log N)$}
{py}{code/DataStructure/Fenwick.py}{}

\Algorithm{세그먼트 트리}
{}{$O(\log N)$}
{cpp}{code/DataStructure/SegmentTree.cpp}
{https://youtu.be/Afr0yvd-8bA?si=ToKEZa9FLnN-h8py}

\Algorithm{세그먼트 트리 + 레이지 프로퍼게이션}
{}{$O(\log N)$}
{cpp}{code/DataStructure/SegmentTreeLazy.cpp}
{JusticeHui}

\Algorithm{퍼시스턴트 세그먼트 트리}
{call init(root[0], s, e) before use}{}
{cpp}{code/DataStructure/PersistentSegmentTree.cpp}
{JusticeHui}

\Algorithm{화성 지도 세그}
{}{}
{cpp}{code/DataStructure/0-1Seg.cpp}
{}

\Algorithm{금광 세그(최대 부분합 세그)}
{}{}
{cpp}{code/DataStructure/GoldMineSeg.cpp}
{}

\Algorithm{머지 소트 트리}
{}{$O((\log N)^2)$}
{cpp}{code/DataStructure/MergeSortTree.cpp}
{kks227}

\Algorithm{단조 큐(Monotone Queue)}
{수열을 순회하면서 구간에 대해 최솟값을 찾는다.}{$O(N))$}
{py}{code/DataStructure/MonotoneQueue.py}{}

\TeamnoteImport{Convex Hull Trick (Stack, LineContainer)}
{code/DataStructure/ConvexHullTrick.cpp}

% 240212
% \TeamnoteImport{Bipartite Union Find}{code/DataStructure/BipartiteUF.cpp}

%\TeamnoteImport{Color Processor}{code/DataStructure/ColorProcessor.cpp}

% 240212

%\TeamnoteImport{Kinetic Segment Tree}{code/DataStructure/KineticSegmentTree.cpp}

%\TeamnoteImport{Lazy LiChao Tree}{code/DataStructure/LazyLiChaoTree.cpp}

%\TeamnoteImport{Splay Tree, Link-Cut Tree}{code/DataStructure/Splay-LCT.cpp}

\section{DP}

\Algorithm{$O(N \log N)$ LIS}{}{}{py}{code/DP/LIS.py}{}

\Algorithm{Berlekamp-Massey, 키타마사법}
{}{Berlekamp-Massey: $O(NK + \log mod)$\\키타마사법: $O(K^2 \log N)$}
{py}{code/DP/berlekamp_massey.py}
{https://gist.github.com/koosaga/d4afc4434dbaa348d5bef0d60ac36aa4}

\Algorithm{SOS(Sum Over Subsets) DP}{}{$O(2^N\times N)$}{cpp}{code/DP/SOSDP.cpp}{}

\Algorithm{$O(N \times \max W)$ Subset Sum (Fast Knapsack)}
{}{}
{cpp}{code/DP/FastKnapsack.cpp}
{kactl}

\Algorithm{DP Optimization, Knuth Optimization}
{}{}
{cpp}{code/DP/DP-Optimize.cpp}
{JusticeHui}
\inputminted[]{cpp}{code/DP/Knuth.cpp} % https://blog.myungwoo.kr/98

\Algorithm{Monotone Queue Optimization}
{}{$O(N\log N)$}
{cpp}{code/DP/MonotoneQueueOpt.cpp}
{aeren}

\Algorithm{Slope Trick}
{}{}
{cpp}{code/DP/SlopeTrick.cpp}
{0xC0DEF}


\Algorithm{Aliens Trick}
{}{$O(N^2 \log |W|)$}
{cpp}{code/DP/Aliens2.cpp}
{JusticeHui}

\section{그래프}

\Algorithm{최단 거리 - Floyd Warshall}{}{$O(V^3)$}{cpp}{code/Graph1/Floyd.cpp}{JusticeHui}
\Algorithm{최단 거리 - Dijkstra}{}{$O(E\log E)$}{cpp}{code/Graph1/Dijkstra.cpp}{JusticeHui}
\Algorithm{최단 거리 - Bellman Ford}{}{$O(VE)$}{cpp}{code/Graph1/Bellman.cpp}{JusticeHui}

\Algorithm{유니온 파인드 + 최소 신장 트리(Kruskal)}
{}{UF: 연산마다 $O(log N)$, MST: $O(E \log E)$}
{cpp}{code/Graph1/Kruskal.cpp}
{JusticeHui}
\inputminted[]{py}{code/Graph1/dsu.py}

\Algorithm{Euler Tour}
{}{}
{cpp}{code/Graph2/EulerTour.cpp}
{JusticeHui}

\Algorithm{SCC - Kosaraju}
{}{$O(V+E)$}
{cpp}{code/Graph2/SCC.cpp}
{JusticeHui}

\Algorithm{BCC - Tarjan}
{}{$O(V+E)$}
{cpp}{code/Graph2/BCC.cpp}
{JusticeHui}

\Algorithm{최대 유량 - Dinic}
{}{$O(V^2E)$, 모든 간선의 용량이 1이면 $O(\min(V^{2/3},E^{1/2})E)$}
{cpp}{code/Graph2/Dinic.cpp}
{JusticeHui}

\Algorithm{MCMF}
{}{}
{cpp}{code/Graph2/MCMF.cpp}
{JusticeHui}

\Algorithm{이분 매칭}
{}{$O(VE)$}
{py}{code/Graph2/BipartiteMatching.py}
{}

\Algorithm{이분 매칭 - Hopcroft Karp}
{}{$O(E \sqrt V)$}
{cpp}{code/Graph2/Hopcroft.cpp}
{JusticeHui}

\subsection{플로우 관련 정리}
\begin{itemize}[noitemsep]
    \item 버텍스 커버(vertex cover)의 여집합은 독립 집합(independent set)이다. 따라서 최소 버텍스 커버(minimum vertex cover)의 여집합은 최대 독립 집합(maximum independent set)이다.
    \item DAG에서 최소 경로 커버(minimum path cover)의 크기는 N - (maximum matching)이다.
    \item 쾨닉의 정리: 모든 bipartite graph에서 maximum matching의 크기와 minimum vertex cover의 크기는 동일하다.
    \item 이분 그래프의 maximum independent set의 크기 = V(전체 정점) - Max Matching
    \item 이분 그래프에서 최소 버텍스 컷은 최소 버텍스 커버와 같다.
    \item 딜워스의 정리: Poset(부분 순서 집합)에서 Maximum Anti Chain의 크기와 Minimum Path Cover의 크기는 같다.
\end{itemize}


\TeamnoteImport{$O(V^3)$ Hungarian Method}
{code/Graph2/Hungarian.cpp} %{e-maxx.ru}

\Algorithm{$O(V^3)$ General Matching}
{}{}
{cpp}{code/Graph2/GeneralMatching.cpp}
{JusticeHui}

\Algorithm{$O(V^3)$ Weighted General Matching}
{}{}
{cpp}{code/Graph2/WeightedMatchingShort.cpp}
{}

\Algorithm{안정 결혼 문제}
{안정적인 매칭은 다음 조건을 만족한다.\\
각 집합의 원소는 상대 집합의 원소들에 대한 서로 다른 선호도를 가진다.\\
첫 번째 집합과 두 번째 집합의 원소가 일대일로 매칭되어 있다.\\
첫 번째 집합에 속한 원소 A, B가 두 번째 집합에 속한 원소 C, D와 각각 매칭되어 있을 때 A가 C보다 D를 선호하면서 D가 B보다 A를 선호하는 경우는 존재하지 않는다.}{$O(V^2)$}
{py}{code/Graph2/StableMarriage.py}{}

\Algorithm{최소 공통 조상(LCA)}
{}{전처리 $O(N \log N)$, 쿼리 $O(\log N)$}
{cpp}{code/Graph2/LCA.cpp}
{JusticeHui}

\Algorithm{Heavy Light Decomposition}
{}{전처리 $O(N)$, 쿼리 $O(T(N) \log N)$}
{cpp}{code/Graph2/HLD.cpp}
{JusticeHui}

\Algorithm{센트로이드 트리}
{}{}
{cpp}{code/Graph2/CentroidTree.cpp}
{mym0404}

\section{수학}

\subsection{바닥함수, 천장함수}
$\left\lceil \frac{n}{m}\right\rceil =\left\lfloor \frac{n+m-1}{m}\right\rfloor =\left\lfloor \frac{n-1}{m}\right\rfloor +1$, n은 정수, m은 양의 정수

$\lfloor -x\rfloor = -\lceil x\rceil; \  \lceil -x\rceil = -\lfloor x\rfloor$

$\lfloor\frac{x+m}{n}\rfloor = \lfloor\frac{\lfloor x\rfloor+m}{n}\rfloor; \  \lceil\frac{x+m}{n}\rceil = \lceil\frac{\lceil x\rceil+m}{n}\rceil$\\

\Algorithm{빠른 거듭제곱}
{$a^b \pmod c$를 구하는 함수}{$O(\log b)$}
{py}{code/Math/PowMod.py}
{}

%\Algorithm{Binary GCD, Extend GCD, CRT, Combination}{}{}{cpp}{code/Math/BasicMath.cpp}{JusticeHui}

%\Algorithm{Diophantine}{}{}{cpp}{code/Math/Diophantine.cpp}{Aeren}
\Algorithm{Power Tower}
{}{}
{cpp}{code/Math/PowerTower.cpp}
{}

\Algorithm{Harmonic Lemma}
{$\sum_{i=1}^n \lfloor\frac{n}{i}\rfloor$}{$O(\sqrt n)$}
{py}{code/Math/HarmonicLemma.py}
{}

\TeamnoteImport{FloorSum}
{code/Math/FloorSum.cpp}

%\Algorithm{XOR Basis (XOR Maximization)}{}{}{cpp}{code/Math/xor-basis.cpp}{Aeren}

%\Algorithm{Stern Brocot Tree}{}{}{cpp}{code/Math/SternBrocotTree.cpp}{myungwoo}

%\TeamnoteImport{$O(N^3 \log 1/\epsilon)$ Polynomial Equation}{code/Math/PolyRoot.cpp}

\Algorithm{Linear Sieve}
{}{}
{cpp}{code/Math/LinearSieve.cpp}
{ahgus89}

\Algorithm{확장 유클리드 알고리즘}
{}{$O(\log \max(a,b))$}
{cpp}{code/Math/ExtendedEuclidean.cpp}
{JusticeHui}

\Algorithm{중국인의 나머지 정리}
{}{$O(k \log m)$}
{cpp}{code/Math/ChineseRemainderTheorem.cpp}
{JusticeHui}

\Algorithm{Diophantine}
{}{}
{cpp}{code/Math/Diophantine.cpp}
{Aeren}

\Algorithm{Miller Rabin + Pollard Rho}
{}{}
{py}{code/Math/MillerRabin-PollardRho.py}
{}

\TeamnoteImport{원시근, 이산로그, 이산제곱근}
{code/Math/DiscreteLogSqrt.cpp}

\TeamnoteImport{$O(N^3 \log 1/\epsilon)$ Polynomial Equation}
{code/Math/PolyRoot.cpp}

\Algorithm{가우스 소거법 - RREF, 랭크, 행렬식, 역행렬}
{}{$O(N^3)$}
{cpp}{code/Math/Matrix.cpp}
{JusticeHui}

\Algorithm{다항식 곱셈(FFT)}
{}{$O(N \log N)$}
{cpp}{code/Math/Convolution.cpp}
{JusticeHui}

\Algorithm{이항 계수를 소수로 나눈 나머지(Lucas' theorem)}
{}{전처리 $O(P)$, 쿼리 $O(\log P)$}
{cpp}{code/Math/LucasTheorem.cpp}
{JusticeHui}

\subsection{이항 계수 관련 공식}
 흡수 항등식: $\binom{r}{k}=\frac{r}{k}\binom{r-1}{k-1}$, 정수 $k\neq0$
 
 덧셈 공식: $\binom{r}{k}=\binom{r-1}{k}+\binom{r-1}{k-1}$
 
 $\binom{r}{k}=(-1)^k\binom{k-r-1}{k}$
 
 $\binom{r}{m}\binom{m}{k}=\binom{r}{k}\binom{r-k}{m-k}$

$\sum_{k\le n}\binom{r+k}{k}=\binom{r+n+1}{n}$, 정수 $n$

$\sum_{0\le k\le n}\binom{k}{m}=\binom{n+1}{m+1}$, 정수 $m,n\ge0$

Vandermonde's Identity: $\binom{m+n}{r}=\sum_{k=0}^r\binom{m}{k}\binom{n}{r-k}$\\
m=n, r=n을 대입하면 $\binom{2n}{n}=\sum\limits_{k=0}^n\binom{n}{k}\binom{n}{n-k}=\sum\limits_{k=0}^n\binom{n}{k}^2$

$\sum_k\binom{r}{m+k}\binom{s}{n-k}=\binom{r+s}{m+n}$, 정수 $m, n$

$\sum_k\binom{l}{m+k}\binom{s}{n+k}=\binom{l+s}{l-m+n}$, 정수 $l\ge0$, 정수 m, n

$\sum_k\binom{l}{m+k}\binom{s+k}{n}(-1)^k=(-1)^{l+m}\binom{s-m}{n-l}$, 정수 $l\ge0$, 정수 m, n

$\sum_{k\le l}\binom{l-k}{m}\binom{s}{k-n}(-1)^k=(-1)^{l+m}\binom{s-m-1}{l-m-n}$, 정수 $l,m,n\ge0$

$\sum_{-q\le k\le l}\binom{l-k}{m}\binom{q+k}{n}=\binom{l+q+1}{m+n+1}$, 정수 $m,n\ge0$, 정수 $l+q\ge0$

\Algorithm{Partition Number}
{}{$O(N\sqrt N)$}{cpp}{code/Math/PartitionNumber.cpp}{Aeren}

\Algorithm{De Bruijn Sequence}
{}{}
{cpp}{code/Math/DeBruijnSequence.cpp}
{DeobureoMinkyuParty}

\Algorithm{지그재그 순열의 개수 구하기}
{지그재그 순열: $a_1<a_2>a_3<a_4>...$ 또는 $a_1>a_2<a_3>a_4<...$}{}{py}{code/Math/ZigzagPermutaion.py}{}

%\TeamnoteImport{Primitive Root, Discrete Log/Sqrt}{code/Math/DiscreteLogSqrt.cpp}

%\Algorithm{Power Tower}{}{}{cpp}{code/Math/PowerTower.cpp}{}

%\Algorithm{De Bruijn Sequence}{}{}{cpp}{code/Math/DeBruijnSequence.cpp}{DeobureoMinkyuParty}

%\Algorithm{Simplex / LP Duality}{}{}{cpp}{code/Math/Simplex.cpp}{molamola}

%\noindent \input{code/Math/SimplexExample}

%\TeamnoteImport{Polynomial \& Convolution}{code/Math/Convolution.cpp}

%\Algorithm{Matroid Intersection}{}{}{cpp}{code/Math/MatroidIntersection.cpp}{Aeren}

\section{문자열}

\Algorithm{정규 표현식}
{}{}
{cpp}{code/String/regex.cpp}
{jinhan814}
\begin{itemize}[noitemsep]
    \item \verb|^|x // \verb|^|은 문자열의 시작을 표현하며, x문자로 시작됨을 의미
    \item x\verb|$| // \verb|$|은 문자열의 종료를 표현하며, x문자로 종료됨을 의미
    \item \verb|.|x // \verb|.|은 개행문자 \verb|\n|을 제외한 다른 모든 문자를 의미
    \item x\verb|+| // \verb|+|은 1회 이상 반복을 의미, x문자가 1번 이상 반복됨을 의미 (\verb|{1,}|과 동일)
    \item x\verb|*| // \verb|*|은 0회 이상 반복을 의미, x문자가 0번 이상 반복됨을 의미 (\verb|{0,}|과 동일)
    \item x\verb|?| // \verb|?|은 0 or 1개 문자 매칭 의미, x문자가 존재할 수도 있고 안할수도 있다는 의미 (\verb|{0,1}|과 동일)
    \item x\verb$|$y // \verb$|$은 or를 표현, x 또는 y가 나온다는 의미
    \item \verb|(|\verb|x|\verb|)| // \verb|()|은 그룹을 표현, 괄호로 묶인 패턴을 의미 ((\verb|abc|)\verb|{3}|와 같이 사용해서 \verb|abcabcabc|를 검출하는데 쓰임)
    \item x\verb|{n}| // \verb|{}|은 반복을 의미, x가 n번 반복됨을 의미
    \item x\verb|{n,}| // \verb|{,}|은 반복을 의미, x가 n번 이상 반복됨을 의미
    \item x\verb|{n,m}| // \verb|{}|은 반복을 의미, x가 n번 이상 m번 이하로 반복됨을 의미
    \item \verb|[|xy\verb|]| // \verb|[]|은 x또는 y를 찾는다는 의미, \verb|[a-z0-9]|이면 알파벳 소문자 또는 숫자를 찾는다는 의미
    \item \verb|[^|xy\verb|]| // \verb|[^]|은 not을 의미, x 및 y 를 제외하고 찾는다는 의미
    \item \verb|[|a-z\verb|]| // \verb|[-]|은 a \verb|~| z 를 찾는다는 의미

    \item \verb|\d| // \verb|\d|은 digit으로 숫자를 의미
    \item \verb|\D| // \verb|\D|은 not digit으로 숫자를 제외하고 나머지 다른 문자를 의미
    \item \verb|\s| // \verb|\s|은 space로 공백문자를 의미
    \item \verb|\S| // \verb|\S|은 not space로 공백문자를 제외한 나머지 다른 문자를 의미
    \item \verb|\t| // \verb|\t|은 tap을 의미
    \item \verb|\w| // \verb|\w|은 알파벳 대문자,소문자와 숫자를 의미, \verb|[A-Za-z0-9]|을 의미
    \item \verb|\W| // \verb|\W|은 not \verb|\w|, 즉 \verb|\w|를 제외한 특수문자를 의미

    \item \verb|(?:)| // 캡쳐하지 않는 그룹 생성
\end{itemize}

\Algorithm{문자열 해싱}
{}{build: $O(N)$, get: $O(1)$}
{cpp}{code/String/Hashing.cpp}
{JusticeHui}

\Algorithm{문자열 매칭 - KMP}
{}{GetFail: $O(\vert P\vert)$, $O(\vert S\vert + \vert P\vert)$}
{cpp}{code/String/KMP.cpp}
{JusticeHui}

\Algorithm{가장 긴 팰린드롬 부분 분자열 - Manacher}
{}{$O(N)$}
{cpp}{code/String/Manacher.cpp}
{JusticeHui}

\Algorithm{문자열 매칭 - Z}
{}{$O(N)$}
{cpp}{code/String/Z.cpp}
{JusticeHui}

\Algorithm{Aho-Corasick}
{}{}
{cpp}{code/String/AhoCorasick.cpp}
{edenooo}

\Algorithm{접미사 배열}
{}{$O(N \log N)$}
{cpp}{code/String/SuffixArray.cpp}
{JusticeHui}

%\Algorithm{$O(N \log N)$ SA + LCP}{}{}{cpp}{code/String/SA-LCP.cpp}{jhnah917}

% \TeamnoteImport{$O(N \log N)$ Tandem Repeats}{code/String/TandemRepeat.cpp}

% \Algorithm{Suffix Automaton}{}{}{cpp}{code/String/SuffixAutomaton.cpp}{aeren}

% \Algorithm{Bitset LCS}{}{}{cpp}{code/String/BitsetLCS.cpp}

\TeamnoteImport{All LCS}
{code/String/AllLCS.cpp}

\section{계산 기하}
\Algorithm{2차원 계산 기하 템플릿 + CCW}
{}{}
{cpp}{code/Geometry/Template.cpp}
{JusticeHui}

\Algorithm{360도 각도 정렬}
{}{}
{cpp}{code/Geometry/PolarSort.cpp}
{JusticeHui}

\Algorithm{다각형 넓이}
{}{}
{cpp}{code/Geometry/PolygonArea.cpp}
{JusticeHui}

\Algorithm{선분 교차 판정}
{}{}
{cpp}{code/Geometry/SegmentIntersection.cpp}
{JusticeHui}

\Algorithm{Intersect Series}
{}{}
{cpp}{code/Geometry/Intersect.cpp}
{JusticeHui, kactl}

\Algorithm{Segment Distance, Segment Reflect}
{}{}
{cpp}{code/Geometry/Segment.cpp}
{JusticeHui}

\Algorithm{다각형 내부 판별}
{}{}
{cpp}{code/Geometry/PointInPolygon.cpp}
{JusticeHui}

\Algorithm{볼록 껍질 - Graham Scan}
{}{}
{cpp}{code/Geometry/ConvexHull.cpp}
{JusticeHui}

\Algorithm{가장 먼 두 점 - Rotating Calipers}
{}{}
{cpp}{code/Geometry/Calipers.cpp}
{JusticeHui}

\Algorithm{볼록 다각형 내부 판별}
{}{}
{cpp}{code/Geometry/PointInConvexPolygon.cpp}
{JusticeHui}

\TeamnoteImport{Segment In Polygon}
{code/Geometry/SegmentInPolygon.cpp}

\Algorithm{Polygon Cut, Center, Union}
{}{}
{cpp}{code/Geometry/Polygon.cpp}
{aeren, kactl}

\TeamnoteImport{Polycon Raycast}
{code/Geometry/PolygonRaycast.cpp}

\Algorithm{$O(N \log N)$ Shamos-Hoey}
{n개의 선분 중 서로 교차하는 선분 쌍이 존재하는지 여부를 O(nlogn)에 판별}{}
{cpp}{code/Geometry/ShamosHoey.cpp}
{qwerasdfzxcl}

\Algorithm{$O(N \log N)$ 반평면 교집합}
{}{}
{cpp}{code/Geometry/HPI.cpp}
{DeobureoMinkyuParty & JusticeHui}


\Algorithm{$O(N^2 \log N)$ Bulldozer Trick(Rotating Sweep Line Technique)}
{서로 다른 정렬의 결과가 $O(N^2)$가지임을 보이고, 가능한 모든 정렬 결과를 $O(N^2\log N)$에 순회하는 테크닉}{}
{cpp}{code/Geometry/Bulldozer.cpp}
{JusticeHui}

\Algorithm{$O(N)$ 최소 외접원}
{}{}
{cpp}{code/Geometry/EnclosingCircle.cpp}
{JusticeHui}


\section{Misc}

\Algorithm{삼분 탐색}
{}{}
{cpp}{code/Misc/TernarySearch.cpp}
{JusticeHui}

\Algorithm{Mo's Algorithm}
{}{$O((N+Q)\sqrt N)$}
{cpp}{code/Misc/Mos.cpp}
{JusticeHui}

\Algorithm{C++ 랜덤, GCC 확장, 비트마스킹 트릭}
{}{}
{cpp}{code/Misc/Cpp-Grammer.cpp}
{JusticeHui}

\Algorithm{Floating Point Add (Kahan)}{}{}{cpp}{code/Misc/FloatingPointAdd.cpp}{JusticeHui}

\Algorithm{Fast I/O, Fast Div, Fast Mod}
{}{}
{cpp}{code/Misc/Constant-Optimize.cpp}
{JusticeHui}

\Algorithm{Python Decimal}
{}{}
{py}{code/Misc/PythonDecimal.py}
{https://github.com/tzupengwang/PECaveros/blob/master/codebook/basic/python_trick.py}

\section{Notes}
\subsection{구간별 약수 최대 개수, 최대 소수} % koosaga
\input{code/Notes/Primes}

\subsection{비트 연산}
Some properties of bitwise operations:
\begin{itemize}[noitemsep]
    \item $a|b = a\oplus b + a\&b$
    \item $a\oplus(a\&b) = (a|b)\oplus b \Leftrightarrow b\oplus a\oplus(a\&b) = (a|b) \Leftrightarrow b\oplus(a\&b) = (a|b)\oplus a$
    \item $(a\&b)\oplus(a|b) = a\oplus b$
\end{itemize}

Addition:
\begin{itemize}[noitemsep]
    \item $a+b = a|b + a\&b$
    \item $a+b = a\oplus b + 2(a\&b)$
\end{itemize}

Subtraction:
\begin{itemize}[noitemsep]
    \item $a-b = (a\oplus(a\&b))-((a|b)\oplus a)$
    \item $a-b = ((a|b)\oplus b)-((a|b)\oplus a)$
    \item $a-b = (a\oplus(a\&b))-(b\oplus(a\&b))$
    \item $a-b = ((a|b)\oplus b)-(b\oplus(a\&b))$
\end{itemize}

application % jinhan814
\begin{itemize}[noitemsep]
    \item 모든 비트 켜기, 오른쪽 i개 비트 켜기 : ~0, (1 $<<$ i) - 1
    \item 2의 거듭제곱으로 나머지, 곱셈, 나눗셈 연산 : n \& ((1 $<<$ i) - 1), n $<<$ i, n $>>$ i
    \item -n = $\sim$(n - 1) : n - 1은 LSB의 오른쪽 비트 flip, -n은 LSB의 왼쪽 비트 flip
    \item n의 LSB(가장 오른쪽의 1인 비트) 구하기 : n \& -n
    \item n이 2의 거듭제곱인지 판별 : (n \& n - 1) == 0 또는 (n \& -n) == n
    \item n의 가장 오른쪽 1 끄기 : n \& (n - 1)
    \item n의 가장 오른쪽 0 켜기 : n $|$ (n + 1)
    \item n의 오른쪽 끝의 연속되는 1을 모두 끄기 : n \& (n + 1)
    \item n의 오른쪽 끝의 연속되는 0을 모두 켜기 : n $|$ (n - 1)
    \item iterating subset in decreasing order : i = (i - 1) \& mask
\end{itemize}

\subsection{Triangles/Trigonometry} % from Ryute Teamnote
\input{code/Notes/Triangle}

\subsection{Calculus, Newton's Method}
\input{code/Notes/Calculus}

\subsection{Zeta/Mobius Transform}
\input{code/Notes/ZetaMobius}

\subsection{Generating Function}
\input{code/Notes/GeneratingFunction}

\subsection{Counting}
\input{code/Notes/Counting}

\subsection{Faulhaber's Formula ($\sum_{k=1}^{n} k^c$)}
\input{code/Notes/Faulhaber}

\subsection{About Graph Degree Sequence}
\input{code/Notes/DegreeSequence}

\subsection{Burnside, Grundy, Pick, Hall, Simpson, Area of Quadrangle, Fermat Point, Euler, Pythagorean}
\input{code/Notes/Theorems}

\subsection{About Graph Minimum Cut}
\input{code/Notes/MinCut}

\subsection{Matrix with Graph(Kirchhoff, Tutte, LGV)}
\input{code/Notes/GraphMatrix}

\subsection{About Graph Matching(Graph with $\vert V \vert \leq 500$)}\input{code/Notes/Matching}

\subsection{Checklist}
\input{code/Notes/Checklist}

\end{multicols*}

\end{document}